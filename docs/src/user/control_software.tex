\chapter{Control Software} 

The Xen control software includes the \xend\ node control daemon
(which must be running), the xm command line tools, and the prototype
xensv web interface.

\section{\Xend\ (node control daemon)}
\label{s:xend}

The Xen Daemon (\Xend) performs system management functions related to
virtual machines.  It forms a central point of control for a machine
and can be controlled using an HTTP-based protocol.  \Xend\ must be
running in order to start and manage virtual machines.

\Xend\ must be run as root because it needs access to privileged
system management functions.  A small set of commands may be issued on
the \xend\ command line:

\begin{tabular}{ll}
  \verb!# xend start! & start \xend, if not already running \\
  \verb!# xend stop!  & stop \xend\ if already running       \\
  \verb!# xend restart! & restart \xend\ if running, otherwise start it \\
  % \verb!# xend trace_start! & start \xend, with very detailed debug logging \\
  \verb!# xend status! & indicates \xend\ status by its return code
\end{tabular}

A SysV init script called {\tt xend} is provided to start \xend\ at
boot time.  {\tt make install} installs this script in
\path{/etc/init.d}.  To enable it, you have to make symbolic links in
the appropriate runlevel directories or use the {\tt chkconfig} tool,
where available.

Once \xend\ is running, more sophisticated administration can be done
using the xm tool (see Section~\ref{s:xm}) and the experimental Xensv
web interface (see Section~\ref{s:xensv}).

As \xend\ runs, events will be logged to \path{/var/log/xend.log} and,
if the migration assistant daemon (\path{xfrd}) has been started,
\path{/var/log/xfrd.log}. These may be of use for troubleshooting
problems.

\section{Xm (command line interface)}
\label{s:xm}

The xm tool is the primary tool for managing Xen from the console.
The general format of an xm command line is:

\begin{verbatim}
# xm command [switches] [arguments] [variables]
\end{verbatim}

The available \emph{switches} and \emph{arguments} are dependent on
the \emph{command} chosen.  The \emph{variables} may be set using
declarations of the form {\tt variable=value} and command line
declarations override any of the values in the configuration file
being used, including the standard variables described above and any
custom variables (for instance, the \path{xmdefconfig} file uses a
{\tt vmid} variable).

The available commands are as follows:

\begin{description}
\item[set-mem] Request a domain to adjust its memory footprint.
\item[create] Create a new domain.
\item[destroy] Kill a domain immediately.
\item[list] List running domains.
\item[shutdown] Ask a domain to shutdown.
\item[dmesg] Fetch the Xen (not Linux!) boot output.
\item[consoles] Lists the available consoles.
\item[console] Connect to the console for a domain.
\item[help] Get help on xm commands.
\item[save] Suspend a domain to disk.
\item[restore] Restore a domain from disk.
\item[pause] Pause a domain's execution.
\item[unpause] Un-pause a domain.
\item[pincpu] Pin a domain to a CPU.
\item[bvt] Set BVT scheduler parameters for a domain.
\item[bvt\_ctxallow] Set the BVT context switching allowance for the
  system.
\item[atropos] Set the atropos parameters for a domain.
\item[rrobin] Set the round robin time slice for the system.
\item[info] Get information about the Xen host.
\item[call] Call a \xend\ HTTP API function directly.
\end{description}

For a detailed overview of switches, arguments and variables to each
command try
\begin{quote}
\begin{verbatim}
# xm help command
\end{verbatim}
\end{quote}

\section{Xensv (web control interface)}
\label{s:xensv}

Xensv is the experimental web control interface for managing a Xen
machine.  It can be used to perform some (but not yet all) of the
management tasks that can be done using the xm tool.

It can be started using:
\begin{quote}
  \verb_# xensv start_
\end{quote}
and stopped using:
\begin{quote}
  \verb_# xensv stop_
\end{quote}

By default, Xensv will serve out the web interface on port 8080.  This
can be changed by editing
\path{/usr/lib/python2.3/site-packages/xen/sv/params.py}.

Once Xensv is running, the web interface can be used to create and
manage running domains.
