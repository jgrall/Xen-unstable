\chapter{Memory Management}

\section{Managing Domain Memory}

XenLinux domains have the ability to relinquish/reclaim machine
memory at the request of the administrator or the user of the domain.

\subsection{Setting memory footprints from dom0}

The machine administrator can request that a domain alter its memory
footprint using the \path{xm mem-set} command.  For instance, we can
request that our example ttylinux domain reduce its memory footprint
to 32 megabytes.

\begin{verbatim}
# xm mem-set ttylinux 32
\end{verbatim}

We can now see the result of this in the output of \path{xm list}:

\begin{verbatim}
# xm list
Name              Id  Mem(MB)  CPU  State  Time(s)  Console
Domain-0           0      251    0  r----    172.2        
ttylinux           5       31    0  -b---      4.3    9605
\end{verbatim}

The domain has responded to the request by returning memory to Xen. We
can restore the domain to its original size using the command line:

\begin{verbatim}
# xm mem-set ttylinux 64
\end{verbatim}

\subsection{Setting memory footprints from within a domain}

The virtual file \path{/proc/xen/balloon} allows the owner of a domain
to adjust their own memory footprint.  Reading the file (e.g.\
\path{cat /proc/xen/balloon}) prints out the current memory footprint
of the domain.  Writing the file (e.g.\ \path{echo new\_target >
  /proc/xen/balloon}) requests that the kernel adjust the domain's
memory footprint to a new value.

\subsection{Setting memory limits}

Xen associates a memory size limit with each domain.  By default, this
is the amount of memory the domain is originally started with,
preventing the domain from ever growing beyond this size.  To permit a
domain to grow beyond its original allocation or to prevent a domain
you've shrunk from reclaiming the memory it relinquished, use the
\path{xm maxmem} command.
