\chapter{Domain Management Tools}

The previous chapter described a simple example of how to configure
and start a domain.  This chapter summarises the tools available to
manage running domains.


\section{Command-line Management}

Command line management tasks are also performed using the \path{xm}
tool.  For online help for the commands available, type:
\begin{quote}
  \verb_# xm help_
\end{quote}

You can also type \path{xm help $<$command$>$} for more information on
a given command.

\subsection{Basic Management Commands}

The most important \path{xm} commands are:
\begin{quote}
  \verb_# xm list_: Lists all domains running.\\
  \verb_# xm consoles_: Gives information about the domain consoles.\\
  \verb_# xm console_: Opens a console to a domain (e.g.\
  \verb_# xm console myVM_)
\end{quote}

\subsection{\tt xm list}

The output of \path{xm list} is in rows of the following format:
\begin{center} {\tt name domid memory cpu state cputime console}
\end{center}

\begin{quote}
  \begin{description}
  \item[name] The descriptive name of the virtual machine.
  \item[domid] The number of the domain ID this virtual machine is
    running in.
  \item[memory] Memory size in megabytes.
  \item[cpu] The CPU this domain is running on.
  \item[state] Domain state consists of 5 fields:
    \begin{description}
    \item[r] running
    \item[b] blocked
    \item[p] paused
    \item[s] shutdown
    \item[c] crashed
    \end{description}
  \item[cputime] How much CPU time (in seconds) the domain has used so
    far.
  \item[console] TCP port accepting connections to the domain's
    console.
  \end{description}
\end{quote}

The \path{xm list} command also supports a long output format when the
\path{-l} switch is used.  This outputs the fulls details of the
running domains in \xend's SXP configuration format.

For example, suppose the system is running the ttylinux domain as
described earlier.  The list command should produce output somewhat
like the following:
\begin{verbatim}
# xm list
Name              Id  Mem(MB)  CPU  State  Time(s)  Console
Domain-0           0      251    0  r----    172.2        
ttylinux           5       63    0  -b---      3.0    9605
\end{verbatim}

Here we can see the details for the ttylinux domain, as well as for
domain~0 (which, of course, is always running).  Note that the console
port for the ttylinux domain is 9605.  This can be connected to by TCP
using a terminal program (e.g. \path{telnet} or, better,
\path{xencons}).  The simplest way to connect is to use the
\path{xm~console} command, specifying the domain name or ID.  To
connect to the console of the ttylinux domain, we could use any of the
following:
\begin{verbatim}
# xm console ttylinux
# xm console 5
# xencons localhost 9605
\end{verbatim}

\section{Domain Save and Restore}

The administrator of a Xen system may suspend a virtual machine's
current state into a disk file in domain~0, allowing it to be resumed
at a later time.

The ttylinux domain described earlier can be suspended to disk using
the command:
\begin{verbatim}
# xm save ttylinux ttylinux.xen
\end{verbatim}

This will stop the domain named `ttylinux' and save its current state
into a file called \path{ttylinux.xen}.

To resume execution of this domain, use the \path{xm restore} command:
\begin{verbatim}
# xm restore ttylinux.xen
\end{verbatim}

This will restore the state of the domain and restart it.  The domain
will carry on as before and the console may be reconnected using the
\path{xm console} command, as above.

\section{Live Migration}

Live migration is used to transfer a domain between physical hosts
whilst that domain continues to perform its usual activities --- from
the user's perspective, the migration should be imperceptible.

To perform a live migration, both hosts must be running Xen / \xend\
and the destination host must have sufficient resources (e.g.\ memory
capacity) to accommodate the domain after the move. Furthermore we
currently require both source and destination machines to be on the
same L2 subnet.

Currently, there is no support for providing automatic remote access
to filesystems stored on local disk when a domain is migrated.
Administrators should choose an appropriate storage solution (i.e.\
SAN, NAS, etc.) to ensure that domain filesystems are also available
on their destination node. GNBD is a good method for exporting a
volume from one machine to another. iSCSI can do a similar job, but is
more complex to set up.

When a domain migrates, it's MAC and IP address move with it, thus it
is only possible to migrate VMs within the same layer-2 network and IP
subnet. If the destination node is on a different subnet, the
administrator would need to manually configure a suitable etherip or
IP tunnel in the domain~0 of the remote node.

A domain may be migrated using the \path{xm migrate} command.  To live
migrate a domain to another machine, we would use the command:

\begin{verbatim}
# xm migrate --live mydomain destination.ournetwork.com
\end{verbatim}

Without the \path{--live} flag, \xend\ simply stops the domain and
copies the memory image over to the new node and restarts it. Since
domains can have large allocations this can be quite time consuming,
even on a Gigabit network. With the \path{--live} flag \xend\ attempts
to keep the domain running while the migration is in progress,
resulting in typical `downtimes' of just 60--300ms.

For now it will be necessary to reconnect to the domain's console on
the new machine using the \path{xm console} command.  If a migrated
domain has any open network connections then they will be preserved,
so SSH connections do not have this limitation.


\section{Managing Domain Memory}

XenLinux domains have the ability to relinquish / reclaim machine
memory at the request of the administrator or the user of the domain.

\subsection{Setting memory footprints from dom0}

The machine administrator can request that a domain alter its memory
footprint using the \path{xm set-mem} command.  For instance, we can
request that our example ttylinux domain reduce its memory footprint
to 32 megabytes.

\begin{verbatim}
# xm set-mem ttylinux 32
\end{verbatim}

We can now see the result of this in the output of \path{xm list}:

\begin{verbatim}
# xm list
Name              Id  Mem(MB)  CPU  State  Time(s)  Console
Domain-0           0      251    0  r----    172.2        
ttylinux           5       31    0  -b---      4.3    9605
\end{verbatim}

The domain has responded to the request by returning memory to Xen. We
can restore the domain to its original size using the command line:

\begin{verbatim}
# xm set-mem ttylinux 64
\end{verbatim}

\subsection{Setting memory footprints from within a domain}

The virtual file \path{/proc/xen/balloon} allows the owner of a domain
to adjust their own memory footprint.  Reading the file (e.g.\
\path{cat /proc/xen/balloon}) prints out the current memory footprint
of the domain.  Writing the file (e.g.\ \path{echo new\_target >
  /proc/xen/balloon}) requests that the kernel adjust the domain's
memory footprint to a new value.

\subsection{Setting memory limits}

Xen associates a memory size limit with each domain.  By default, this
is the amount of memory the domain is originally started with,
preventing the domain from ever growing beyond this size.  To permit a
domain to grow beyond its original allocation or to prevent a domain
you've shrunk from reclaiming the memory it relinquished, use the
\path{xm maxmem} command.
