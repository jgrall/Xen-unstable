\chapter{Installation}

The Xen distribution includes three main components: Xen itself, ports
of Linux and NetBSD to run on Xen, and the userspace
tools required to manage a Xen-based system.  This chapter describes
how to install the Xen~2.0 distribution from source.  Alternatively,
there may be pre-built packages available as part of your operating
system distribution.


\section{Prerequisites}
\label{sec:prerequisites}

The following is a full list of prerequisites.  Items marked `$\dag$'
are required by the \xend\ control tools, and hence required if you
want to run more than one virtual machine; items marked `$*$' are only
required if you wish to build from source.
\begin{itemize}
\item A working Linux distribution using the GRUB bootloader and
  running on a P6-class (or newer) CPU.
\item [$\dag$] The \path{iproute2} package.
\item [$\dag$] The Linux bridge-utils\footnote{Available from {\tt
      http://bridge.sourceforge.net}} (e.g., \path{/sbin/brctl})
\item [$\dag$] The Linux hotplug system\footnote{Available from {\tt
      http://linux-hotplug.sourceforge.net/}} (e.g., \path{/sbin/hotplug}
      and related scripts)
\item [$\dag$] An installation of Twisted~v1.3 or
  above\footnote{Available from {\tt http://www.twistedmatrix.com}}.
  There may be a binary package available for your distribution;
  alternatively it can be installed by running `{\sl make
    install-twisted}' in the root of the Xen source tree.
\item [$*$] Build tools (gcc v3.2.x or v3.3.x, binutils, GNU make).
\item [$*$] Development installation of libcurl (e.g., libcurl-devel)
\item [$*$] Development installation of zlib (e.g., zlib-dev).
\item [$*$] Development installation of Python v2.2 or later (e.g.,
  python-dev).
\item [$*$] \LaTeX\ and transfig are required to build the
  documentation.
\end{itemize}

Once you have satisfied the relevant prerequisites, you can now
install either a binary or source distribution of Xen.


\section{Installing from Binary Tarball}

Pre-built tarballs are available for download from the Xen download
page
\begin{quote} {\tt http://xen.sf.net}
\end{quote}

Once you've downloaded the tarball, simply unpack and install:
\begin{verbatim}
# tar zxvf xen-2.0-install.tgz
# cd xen-2.0-install
# sh ./install.sh
\end{verbatim}

Once you've installed the binaries you need to configure your system
as described in Section~\ref{s:configure}.


\section{Installing from Source}

This section describes how to obtain, build, and install Xen from
source.

\subsection{Obtaining the Source}

The Xen source tree is available as either a compressed source tar
ball or as a clone of our master BitKeeper repository.

\begin{description}
\item[Obtaining the Source Tarball]\mbox{} \\
  Stable versions (and daily snapshots) of the Xen source tree are
  available as compressed tarballs from the Xen download page
  \begin{quote} {\tt http://xen.sf.net}
  \end{quote}

\item[Using BitKeeper]\mbox{} \\
  If you wish to install Xen from a clone of our latest BitKeeper
  repository then you will need to install the BitKeeper tools.
  Download instructions for BitKeeper can be obtained by filling out
  the form at:
  \begin{quote} {\tt http://www.bitmover.com/cgi-bin/download.cgi}
\end{quote}
The public master BK repository for the 2.0 release lives at:
\begin{quote} {\tt bk://xen.bkbits.net/xen-2.0.bk}
\end{quote} 
You can use BitKeeper to download it and keep it updated with the
latest features and fixes.

Change to the directory in which you want to put the source code, then
run:
\begin{verbatim}
# bk clone bk://xen.bkbits.net/xen-2.0.bk
\end{verbatim}

Under your current directory, a new directory named \path{xen-2.0.bk}
has been created, which contains all the source code for Xen, the OS
ports, and the control tools. You can update your repository with the
latest changes at any time by running:
\begin{verbatim}
# cd xen-2.0.bk # to change into the local repository
# bk pull       # to update the repository
\end{verbatim}
\end{description}

% \section{The distribution}
%
% The Xen source code repository is structured as follows:
%
% \begin{description}
% \item[\path{tools/}] Xen node controller daemon (Xend), command line
%   tools, control libraries
% \item[\path{xen/}] The Xen VMM.
% \item[\path{linux-*-xen-sparse/}] Xen support for Linux.
% \item[\path{linux-*-patches/}] Experimental patches for Linux.
% \item[\path{netbsd-*-xen-sparse/}] Xen support for NetBSD.
% \item[\path{docs/}] Various documentation files for users and
%   developers.
% \item[\path{extras/}] Bonus extras.
% \end{description}

\subsection{Building from Source}

The top-level Xen Makefile includes a target `world' that will do the
following:

\begin{itemize}
\item Build Xen.
\item Build the control tools, including \xend.
\item Download (if necessary) and unpack the Linux 2.6 source code,
  and patch it for use with Xen.
\item Build a Linux kernel to use in domain 0 and a smaller
  unprivileged kernel, which can optionally be used for unprivileged
  virtual machines.
\end{itemize}

After the build has completed you should have a top-level directory
called \path{dist/} in which all resulting targets will be placed; of
particular interest are the two kernels XenLinux kernel images, one
with a `-xen0' extension which contains hardware device drivers and
drivers for Xen's virtual devices, and one with a `-xenU' extension
that just contains the virtual ones. These are found in
\path{dist/install/boot/} along with the image for Xen itself and the
configuration files used during the build.

The NetBSD port can be built using:
\begin{quote}
\begin{verbatim}
# make netbsd20
\end{verbatim}
\end{quote}
NetBSD port is built using a snapshot of the netbsd-2-0 cvs branch.
The snapshot is downloaded as part of the build process, if it is not
yet present in the \path{NETBSD\_SRC\_PATH} search path.  The build
process also downloads a toolchain which includes all the tools
necessary to build the NetBSD kernel under Linux.

To customize further the set of kernels built you need to edit the
top-level Makefile. Look for the line:

\begin{quote}
\begin{verbatim}
KERNELS ?= mk.linux-2.6-xen0 mk.linux-2.6-xenU
\end{verbatim}
\end{quote}

You can edit this line to include any set of operating system kernels
which have configurations in the top-level \path{buildconfigs/}
directory, for example \path{mk.linux-2.6-xenU} to build a Linux 2.6
kernel containing only virtual device drivers.

%% Inspect the Makefile if you want to see what goes on during a
%% build.  Building Xen and the tools is straightforward, but XenLinux
%% is more complicated.  The makefile needs a `pristine' Linux kernel
%% tree to which it will then add the Xen architecture files.  You can
%% tell the makefile the location of the appropriate Linux compressed
%% tar file by
%% setting the LINUX\_SRC environment variable, e.g. \\
%% \verb!# LINUX_SRC=/tmp/linux-2.6.11.tar.bz2 make world! \\ or by
%% placing the tar file somewhere in the search path of {\tt
%%   LINUX\_SRC\_PATH} which defaults to `{\tt .:..}'.  If the
%% makefile can't find a suitable kernel tar file it attempts to
%% download it from kernel.org (this won't work if you're behind a
%% firewall).

%% After untaring the pristine kernel tree, the makefile uses the {\tt
%%   mkbuildtree} script to add the Xen patches to the kernel.


%% \framebox{\parbox{5in}{
%%     {\bf Distro specific:} \\
%%     {\it Gentoo} --- if not using udev (most installations,
%%     currently), you'll need to enable devfs and devfs mount at boot
%%     time in the xen0 config.  }}

\subsection{Custom XenLinux Builds}

% If you have an SMP machine you may wish to give the {\tt '-j4'}
% argument to make to get a parallel build.

If you wish to build a customized XenLinux kernel (e.g. to support
additional devices or enable distribution-required features), you can
use the standard Linux configuration mechanisms, specifying that the
architecture being built for is \path{xen}, e.g:
\begin{quote}
\begin{verbatim}
# cd linux-2.6.11-xen0
# make ARCH=xen xconfig
# cd ..
# make
\end{verbatim}
\end{quote}

You can also copy an existing Linux configuration (\path{.config})
into \path{linux-2.6.11-xen0} and execute:
\begin{quote}
\begin{verbatim}
# make ARCH=xen oldconfig
\end{verbatim}
\end{quote}

You may be prompted with some Xen-specific options; we advise
accepting the defaults for these options.

Note that the only difference between the two types of Linux kernel
that are built is the configuration file used for each.  The `U'
suffixed (unprivileged) versions don't contain any of the physical
hardware device drivers, leading to a 30\% reduction in size; hence
you may prefer these for your non-privileged domains.  The `0'
suffixed privileged versions can be used to boot the system, as well
as in driver domains and unprivileged domains.

\subsection{Installing the Binaries}

The files produced by the build process are stored under the
\path{dist/install/} directory. To install them in their default
locations, do:
\begin{quote}
\begin{verbatim}
# make install
\end{verbatim}
\end{quote}

Alternatively, users with special installation requirements may wish
to install them manually by copying the files to their appropriate
destinations.

%% Files in \path{install/boot/} include:
%% \begin{itemize}
%% \item \path{install/boot/xen-2.0.gz} Link to the Xen 'kernel'
%% \item \path{install/boot/vmlinuz-2.6-xen0} Link to domain 0
%%   XenLinux kernel
%% \item \path{install/boot/vmlinuz-2.6-xenU} Link to unprivileged
%%   XenLinux kernel
%% \end{itemize}

The \path{dist/install/boot} directory will also contain the config
files used for building the XenLinux kernels, and also versions of Xen
and XenLinux kernels that contain debug symbols (\path{xen-syms-2.0.6}
and \path{vmlinux-syms-2.6.11.11-xen0}) which are essential for
interpreting crash dumps.  Retain these files as the developers may
wish to see them if you post on the mailing list.


\section{Configuration}
\label{s:configure}

Once you have built and installed the Xen distribution, it is simple
to prepare the machine for booting and running Xen.

\subsection{GRUB Configuration}

An entry should be added to \path{grub.conf} (often found under
\path{/boot/} or \path{/boot/grub/}) to allow Xen / XenLinux to boot.
This file is sometimes called \path{menu.lst}, depending on your
distribution.  The entry should look something like the following:

{\small
\begin{verbatim}
title Xen 2.0 / XenLinux 2.6
  kernel /boot/xen-2.0.gz dom0_mem=131072
  module /boot/vmlinuz-2.6-xen0 root=/dev/sda4 ro console=tty0
\end{verbatim}
}

The kernel line tells GRUB where to find Xen itself and what boot
parameters should be passed to it (in this case, setting domain 0's
memory allocation in kilobytes and the settings for the serial port).
For more details on the various Xen boot parameters see
Section~\ref{s:xboot}.

The module line of the configuration describes the location of the
XenLinux kernel that Xen should start and the parameters that should
be passed to it (these are standard Linux parameters, identifying the
root device and specifying it be initially mounted read only and
instructing that console output be sent to the screen).  Some
distributions such as SuSE do not require the \path{ro} parameter.

%% \framebox{\parbox{5in}{
%%     {\bf Distro specific:} \\
%%     {\it SuSE} --- Omit the {\tt ro} option from the XenLinux
%%     kernel command line, since the partition won't be remounted rw
%%     during boot.  }}


If you want to use an initrd, just add another \path{module} line to
the configuration, as usual:

{\small
\begin{verbatim}
  module /boot/my_initrd.gz
\end{verbatim}
}

As always when installing a new kernel, it is recommended that you do
not delete existing menu options from \path{menu.lst} --- you may want
to boot your old Linux kernel in future, particularly if you have
problems.

\subsection{Serial Console (optional)}

%% kernel /boot/xen-2.0.gz dom0_mem=131072 com1=115200,8n1
%% module /boot/vmlinuz-2.6-xen0 root=/dev/sda4 ro


In order to configure Xen serial console output, it is necessary to
add an boot option to your GRUB config; e.g.\ replace the above kernel
line with:
\begin{quote}
{\small
\begin{verbatim}
   kernel /boot/xen.gz dom0_mem=131072 com1=115200,8n1
\end{verbatim}}
\end{quote}

This configures Xen to output on COM1 at 115,200 baud, 8 data bits, 1
stop bit and no parity. Modify these parameters for your set up.

One can also configure XenLinux to share the serial console; to
achieve this append ``\path{console=ttyS0}'' to your module line.

If you wish to be able to log in over the XenLinux serial console it
is necessary to add a line into \path{/etc/inittab}, just as per
regular Linux. Simply add the line:
\begin{quote} {\small {\tt c:2345:respawn:/sbin/mingetty ttyS0}}
\end{quote}

and you should be able to log in. Note that to successfully log in as
root over the serial line will require adding \path{ttyS0} to
\path{/etc/securetty} in most modern distributions.

\subsection{TLS Libraries}

Users of the XenLinux 2.6 kernel should disable Thread Local Storage
(e.g.\ by doing a \path{mv /lib/tls /lib/tls.disabled}) before
attempting to run with a XenLinux kernel\footnote{If you boot without
  first disabling TLS, you will get a warning message during the boot
  process. In this case, simply perform the rename after the machine
  is up and then run \texttt{/sbin/ldconfig} to make it take effect.}.
You can always reenable it by restoring the directory to its original
location (i.e.\ \path{mv /lib/tls.disabled /lib/tls}).

The reason for this is that the current TLS implementation uses
segmentation in a way that is not permissible under Xen.  If TLS is
not disabled, an emulation mode is used within Xen which reduces
performance substantially.

We hope that this issue can be resolved by working with Linux
distribution vendors to implement a minor backward-compatible change
to the TLS library.


\section{Booting Xen}

It should now be possible to restart the system and use Xen.  Reboot
as usual but choose the new Xen option when the Grub screen appears.

What follows should look much like a conventional Linux boot.  The
first portion of the output comes from Xen itself, supplying low level
information about itself and the machine it is running on.  The
following portion of the output comes from XenLinux.

You may see some errors during the XenLinux boot.  These are not
necessarily anything to worry about --- they may result from kernel
configuration differences between your XenLinux kernel and the one you
usually use.

When the boot completes, you should be able to log into your system as
usual.  If you are unable to log in to your system running Xen, you
should still be able to reboot with your normal Linux kernel.
