%
% Copyright (c) 2006 XenSource, Inc.
%
% Permission is granted to copy, distribute and/or modify this document under
% the terms of the GNU Free Documentation License, Version 1.2 or any later
% version published by the Free Software Foundation; with no Invariant
% Sections, no Front-Cover Texts and no Back-Cover Texts.  A copy of the
% license is included in the section entitled
% "GNU Free Documentation License" or the file fdl.tex.
%
% Authors: Ewan Mellor, Richard Sharp, Dave Scott, Jon Harrop.
%

\section{Wire Protocol for Remote API Calls}

API calls are sent over a network to a Xen-enabled host using
the XML-RPC protocol. In this Section we describe how the
higher-level types used in our API Reference are mapped to
primitive XML-RPC types.

In our API Reference we specify the signatures of API functions in the following
style:
\begin{verbatim}
    (ref_vm Set)   Host.ListAllVMs()
\end{verbatim}
This specifies that the function with name {\tt Host.ListAllVMs} takes
no parameters and returns a Set of {\tt ref\_vm}s.
These types are mapped onto XML-RPC types in a straight-forward manner:
\begin{itemize}
  \item Floats, Bools, DateTimes and Strings map directly to the XML-RPC {\tt
  double}, {\tt boolean}, {\tt dateTime.iso8601}, and {\tt string} elements.

  \item all our ``{\tt ref\_}'' types (e.g.\ {\tt ref\_vm} in the above
  example) map to XML-RPC's {\tt String} type.  The string itself is the OSF
  DCE UUID presentation format (as output by {\tt uuidgen}, etc).

  \item ints are all assumed to be 64-bit in our API and are encoded as a
  string of decimal digits (rather than using XML-RPC's built-in 32-bit {\tt
  i4} type).

  \item values of enum types are encoded as strings. For example, a value of
  {\tt destroy} of type {\tt on\_normal\_exit}, would be conveyed as:
  \begin{verbatim}
    <value><string>destroy</string></value>
  \end{verbatim}

  \item for all our types, {\tt t}, our type {\tt t Set} simply maps to
  XML-RPC's {\tt Array} type, so for example a value of type {\tt cpu\_feature
  Set} would be transmitted like this:

  \begin{verbatim}
<array>
  <data>
    <value><string>CX8</string></value>
    <value><string>PSE36</string></value>
    <value><string>FPU</string></value>
  </data>
</array> 
  \end{verbatim}

  \item for types {\tt k} and {\tt v}, our type {\tt (k, v) Map} maps onto an
  XML-RPC struct, with the key as the name of the struct.  Note that the {\tt
  (k, v) Map} type is only valid when {\tt k} is a {\tt String}, {\tt Ref}, or
  {\tt Int}, and in each case the keys of the maps are stringified as
  above. For example, the {\tt (String, double) Map} containing a the mappings
  Mike $\rightarrow$ 2.3 and John $\rightarrow$ 1.2 would be represented as:

  \begin{verbatim}
<value>
  <struct>
    <member>
      <name>Mike</name>
      <value><double>2.3</double></value>
    </member>
    <member>
      <name>John</name>
      <value><double>1.2</double></value>
    </member>
  </struct>
</value>
  \end{verbatim}

  \item our {\tt Void} type is transmitted as an empty string.

\end{itemize}

\subsection{Return Values/Status Codes}
\label{synchronous-result}

The return value of an RPC call is an XML-RPC {\tt Struct}.

\begin{itemize}
\item The first element of the struct is named {\tt Status}; it
contains a string value indicating whether the result of the call was
a ``{\tt Success}'' or a ``{\tt Failure}''.
\end{itemize}

If {\tt Status} was set to {\tt Success} then the Struct contains a second
element named {\tt Value}:
\begin{itemize}
\item The element of the struct named {\tt Value} contains the function's return value.
\end{itemize}

In the case where {\tt Status} is set to {\tt Failure} then
the struct contains a second element named {\tt ErrorDescription}:
\begin{itemize}
\item The element of the struct named {\tt ErrorDescription} contains
an array of string values. The first element of the array represents an error code;
the remainder of the array represents error parameters relating to that code.
\end{itemize}

For example, an XML-RPC return value from the {\tt Host.ListAllVMs} function above
may look like this:
\begin{verbatim}
    <struct>
       <member>
         <name>Status</name>
         <value>Success</value>
       </member>
       <member>
          <name>Value</name>
          <value>
            <array>
               <data>
                 <value>vm-id-1</value>
                 <value>vm-id-2</value>
                 <value>vm-id-3</value>
               </data>
            </array>
         </value>
       </member>
    </struct>
\end{verbatim}

\section{Making XML-RPC Calls}

\subsection{Transport Layer}

We ought to support at least
\begin{itemize}
\item HTTP/S for remote administration
\item HTTP over Unix domain sockets for local administration
\end{itemize}

\subsection{Session Layer}

The XML-RPC interface is session-based; before you can make arbitrary RPC calls
you must login and initiate a session. For example:
\begin{verbatim}
   session_id    Session.login_with_password(string uname, string pwd)
\end{verbatim}
Where {\tt uname} and {\tt password} refer to your username and password
respectively, as defined by the Xen administrator.
The {\tt session\_id} returned by {\tt Session.Login} is passed to subequent
RPC calls as an authentication token.

A session can be terminated with the {\tt Session.Logout} function:
\begin{verbatim}
   void          Session.Logout(session_id session)
\end{verbatim}

\subsection{Synchronous and Asynchronous invocation}

Each method call (apart from those on ``Session'' and ``Task'' objects)
can be made either synchronously or asynchronously.
A synchronous RPC call blocks until the
return value is received; the return value of a synchronous RPC call is
exactly as specified in Section~\ref{synchronous-result}.

Each of the methods specified in the API Reference is synchronous.
However, although not listed explicitly in this document, each
method call has an asynchronous analogue in the {\tt Async}
namespace. For example, synchronous call {\tt VM.Install(...)}
(described in Chapter~\ref{api-reference})
has an asynchronous counterpart, {\tt
Async.VM.Install(...)}, that is non-blocking.

Instead of returning its result directly, an asynchronous RPC call
returns a {\tt task-id}; this identifier is subsequently used
to track the status of a running asynchronous RPC. Note that an asychronous
call may fail immediately, before a {\tt task-id} has even been created---to
represent this eventuality, the returned {\tt task-id}
is wrapped in an XML-RPC struct with a {\tt Status}, {\tt ErrorDescription} and
{\tt Value} fields, exactly as specified in Section~\ref{synchronous-result}.

The {\tt task-id} is provided in the {\tt Value} field if {\tt Status} is set to
{\tt Success}.

Two special RPC calls are provided to poll the status of
asynchronous calls:
\begin{verbatim}
    Array<task_id>  Async.Task.GetAllTasks (session_id s)
    task_status     Async.Task.GetStatus   (session_id s, task_id t)
\end{verbatim}

{\tt Async.Task.GetAllTasks} returns a set of the currently
executing asynchronous tasks belong to the current user\footnote{
%
The current user is determined by the username that was provided
to {\tt Session.Login}.
%
}.

{\tt Async.Task.GetStatus} returns a {\tt task\_status} result.
This is an XML-RPC struct with three elements:
\begin{itemize}
  \item The first element is named {\tt Progress} and contains
an {\tt Integer} between 0 and 100 representing the estimated percentage of
the task currently completed.
  \item The second element is named {\tt ETA} and contains a {\tt DateTime} 
representing the estimated time the task will be complete.
  \item The third element is named {\tt Result}. If {\tt Progress}
is not 100 then {\tt Result} contains the empty string. If {\tt Progress}
{\em is\/} set to 100, then {\tt Result} contains the function's return
result (as specified in Section~\ref{synchronous-result})\footnote{
%
Recall that this itself is a struct potentially containing status, errorcode,
value fields etc.
%
}.
\end{itemize}

\section{Example interactive session}

This section describes how an interactive session might look, using the python
XML-RPC client library. 

First, initialise python and import the library {\tt xmlrpclib}:

\begin{verbatim}
\$ python2.4
...
>>> import xmlrpclib
\end{verbatim}

Create a python object referencing the remote server:

\begin{verbatim}
>>> xen = xmlrpclib.Server("http://test:4464")
\end{verbatim}

Acquire a session token by logging in with a username and password (error-handling ommitted for brevity; the session token is pointed to by the key {\tt 'Value'} in the returned dictionary)

\begin{verbatim}
>>> session = xen.Session.do_login_with_password("user", "passwd")['Value']
\end{verbatim}

When serialised, this call looks like the following:

\begin{verbatim}
<?xml version='1.0'?>
<methodCall>
  <methodName>Session.do_login_with_password</methodName>
  <params>
    <param>
      <value><string>user</string></value>
    </param>
    <param>
      <value><string>passwd</string></value>
    </param>
  </params>
</methodCall>
\end{verbatim}

Next, the user may acquire a list of all the VMs known to the host: (Note the call takes the session token as the only parameter)

\begin{verbatim}
>>> all_vms = xen.VM.do_list(session)['Value']
>>> all_vms
['b7b92d9e-d442-4710-92a5-ab039fd7d89b', '23e1e837-abbf-4675-b077-d4007989b0cc', '2045dbc0-0734-4eea-9cb2-b8218c6b5bf2', '3202ae18-a046-4c32-9fda-e32e9631866e']
\end{verbatim}

Note the VM references are internally UUIDs. Once a reference to a VM has been acquired a lifecycle operation may be invoked:

\begin{verbatim}
>>> xen.VM.do_start(session, all_vms[3], False)
{'Status': 'Failure', 'ErrorDescription': 'Operation not implemented'}
\end{verbatim}

In this case the {\tt start} message has not been implemented and an error response has been returned. Currently these high-level errors are returned as structured data (rather than as XMLRPC faults), allowing for internationalised errors in future. Finally, here are some examples of using accessors for object fields:

\begin{verbatim}
>>> xen.VM.getname_label(session, all_vms[3])['Value']
'SMP'
>>> xen.VM.getname_description(session, all_vms[3])['Value']
'Debian for Xen'
\end{verbatim}
